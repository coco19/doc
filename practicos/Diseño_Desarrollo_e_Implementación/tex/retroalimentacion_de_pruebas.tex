\section{Ejecución, documentación y retroalimentación de pruebas}
%Sólo se debe registrar en el mismo formulario o documento del plan de pruebas de la etapa anterior. Se deben registrar los resultados obtenidos por la ejecución de las pruebas planificadas y las acciones correctivas realizadas, si hubieren correspondido (tengan en cuenta lo aportado por el Ing. Diego Villa).
 La retroalimentación de las pruebas para la funcionalidad del sistema que se refiere a ``carga y muestra de análisis'', se detallan al finalizar los sprint que se describen a continuación, dichos sprint se encuentran acompañados de una breve descripción de las actividades realizadas en cada uno de ellos.
 \begin{itemize}
 	\item \textbf{Sprint 2:} Desarrollo de la carga y muestra de mediciones, se realizaron las pruebas necesarios y se hizo la retroalimentación respectiva.
 	\item \textbf{Sprint 3:} Corrección de issues detectados por las pruebas realizadas en el sprint 2.
 	\item \textbf{Sprint 4:} Aquí se desarrolló la presentación en forma gráfica de las mediciones, se indicaron las pruebas y la retroalimentación de las mismas.
 	\item \textbf{Sprint 5:} En este sprint se desarrollo la funcionalidad de autenticación del usuario, si bien esta función no se refiere a mediciones, todo el sistema hace uso de la misma y ya sea para cargar o ver la información, el usuario debe validarse.
 	\item \textbf{Sprint 6: } Este Sprint junto al \textbf{Sprint 2} y \textbf{Sprint 7}, es el mas importante, ya que aquí se desarrolló completamente la funcionalidad que le permite cargar un análisis completo, con imágenes y mediciones,  y que luego pueda verlo de diferentes maneras, ya sea de forma gráfica, resumida o tabular.
 	\item \textbf{Sprint 7: } Aquí se desarrollo la funcionalidad le permite al usuario cargar un archivo y/o imagen, YesDoc guarda estos documentos en una cuenta de dropbox solo accedida por el usuario.
 \end{itemize}
 
 Para facilitar la lectura, a continuación se detallará una síntesis del resultado de las pruebas realizadas en cada uno de los sprint.
 
 	\begin{itemize}
 		\item \textbf{¿Qué fue bien?}
 		\begin{itemize}
 			\item        Las cargas y ediciones de las mediciones se llevan a cabo correctamente.		
			\item        Las cargas y muestra de mediciones en forma gráfica se llevan a cabo correctamente.			
			\item La carga de análisis, junto a las mediciones e imágenes funciona correctamente
			\item Listar los análisis y su contenido funcionan bien en Firefox.
			\item Se definió correctamente la gestión de análisis y el uso de autenticación para la gestión de análisis personales.
			\item Se armaron las vistas que permiten al usuario crear análisis y cargar mediciones e imágenes a los mismos.			
			\item La authorización a través del servicio de dropbox se lleva a cabo correctamente.
			\item Se realizan correctamente tanto la carga como la eliminación de archivos de dropbox.			
 		\end{itemize}
 		
 		\item \textbf{¿Qué se mejoró?}
 		\begin{itemize}
 			\item \textbf{Cerrado} Al crear una nueva medición, se mostraba un cartel (alert de javascript) con una fecha, dicho alert fue eliminado.
 			\item \textbf{Cerrado} Se encontró un problema con la zona horaria que usa el servidor y la zona horaria del usuario, para solucionarlo hubo q hacer un casteo previo cuando se solicitaba la fecha y hora del usuario para mostrar.
			\item \textbf{Cerrado en Sprint 8} Las imágenes de los análisis no es posible verlo en el navegador Chrome.
			\item \textbf{Cerrado} Los colores e imágenes de los botones se cambiaron por unos mas representativos.
			\item \textbf{Cerrado} Se permite hacer zoom en las gráficas.
			\item Se mejoró la captura de mediciones permitiendo que estas se engloben en el contexto de un análisis lo cual les da una razón de ser.
            \item \textbf{Cerrado} Para el problema con el SDK de dropbox se implementa el request sin hacer uso del SDK. El thumbnail se solicita con un tamaño de 640x480.
            \item \textbf{Cerrado} Se corrige la gestión de las claves de la aplicación en Dropbox, incluyendo la clave pública e importando la clave privada desde la variable de entorno DROPBOX\_APP\_SECRET.
            \item \textbf{Cerrado} Se corrige la devolución de archivos cargados en Dropbox, y se deja este servicio de almacenamiento como predeterminado, en caso de que el usuario no cuente con un servicio asociado.
            \item \textbf{Cerrado} La carga de archivos se hacía a través de un modelo Epicrisis el cual se cambio por el modelo Análisis que resulta más representativo.
            \item \textbf{Cerrado} Se integró la carga de archivos como el servicio POST del recurso AnlysisFileList.	
			\item \textbf{Cerrado en sprint 8} En el futuro se deberá mejorar las validaciones de los datos a la hora de cargar información en los formularios.   
 			\item \textbf{Cerrado sprint 6} Solo debería mostrarse las unidades relacionadas al tipo de medición que se ha seleccionado	
 			\item \textbf{Cerrado sprint 9} La posición de la botonera en la sección histórico no es la correcta.		         										
 			\item \textbf{Cerrado sprint 8} Al cargar un nuevo usuario y al modificarlo, el formulario muestra errores, esto produce desconcierto en el usuario.
		        \item \textbf{Cerrado} Se desarrolló la opción de Google drive como otro medio de almacenamiento en la nube.
		        \item \textbf{Cerrado } Se muestran carteles de advertencia cuando el usuario selecciona en eliminar algo.
 		\end{itemize}
 		
 		\item \textbf{¿Qué se puede mejorar?}
 		\begin{itemize}
 
 			\item \textbf{Abierto} Se deberá mejorar la manera de seleccionar la fecha y la hora.
 			\item \textbf{Abierto} Deberá realizarse los carteles de advertencia necesarios.
		    \item \textbf{Cerrado} En el futuro se deberá mejorar las validaciones de los datos a la hora de cargar información en los formularios. 
			 \item \textbf{Abierto}  El aviso de medición cargada debería ser mas nítido.
			 \item Se debe mejorar en cuanto a los tiempos de trabajo del back end para brindar soluciones que puedan ser usadas a tiempo por el front end.
			 \item En cuanto a la gestión de análisis se podría definir que cuando el usuario cargue su primer medida del día, sin indicar un análisis específico, se cree automáticamente un análisis con descripción ``Análisis diario'' si es que éste aún no existe.	
	          \item \textbf{Abierto} En la versión actual de dropbox/dropbox-sdk-python (3.37), el método para obtener el thumbnail de una imagen no funciona, debido a problemas con ThumbnailSize.			 		 		    
		        \item \textbf{Abierto} En el futuro se deberá desarrollar opciones para otros medios de almacenamiento en la nube como pueden ser Google Drive o Mega.	          
 		\end{itemize}
 		
 		
 	\end{itemize}